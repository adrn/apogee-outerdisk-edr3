% # To-dos
% - See comments labeled "TODO" below
% -

\documentclass[modern]{aastex63}
\usepackage[ruled,vlined,norelsize]{algorithm2e}
\usepackage[framemethod=tikz]{mdframed}
\usetikzlibrary{shadows}
\definecolor{captiongray}{HTML}{555555}
\mdfsetup{%
  innertopmargin=2ex,
  innerbottommargin=1.8ex,
  linecolor=captiongray,
  linewidth=0.5pt,
  roundcorner=5pt,
  shadow=false
}
\usepackage{enumitem}
\usepackage{booktabs}
\usepackage{xcolor}
\usepackage{amsmath}

\newcommand{\documentname}{\textsl{Article}}
\newcommand{\sectionname}{Section}
\renewcommand{\figurename}{Figure}
\newcommand{\equationname}{Equation}
\renewcommand{\tablename}{Table}

% Misc.
\newcommand{\bs}[1]{\boldsymbol{#1}}

% Missions
\newcommand{\project}[1]{\textsl{#1}}

% Packages / projects / programming
\newcommand{\package}[1]{\textsl{#1}}
\newcommand{\acronym}[1]{{\small{#1}}}
\newcommand{\github}{\package{GitHub}}
\newcommand{\python}{\package{Python}}

% Stats / probability
\newcommand{\given}{\,|\,}
\newcommand{\norm}{\mathcal{N}}
\newcommand{\pdf}{\textsl{pdf}}

% Maths
\newcommand{\dd}{\mathrm{d}}
\newcommand{\TT}[1]{\ensuremath{{#1}^{\mathsf{T}}}}
\newcommand{\transp}{\ensuremath{^{\mathsf{T}}}}
\newcommand{\inv}[1]{{#1}^{-1}}
\newcommand{\argmin}{\operatornamewithlimits{argmin}}
\newcommand{\mean}[1]{\left\langle #1 \right\rangle}

% Non-scalar variables
\renewcommand{\vec}[1]{\ensuremath{\bs{#1}}}
\newcommand{\mat}[1]{\ensuremath{\mathbf{#1}}}

% Unit shortcuts
\newcommand{\Msun}{\ensuremath{\mathrm{M}_\odot}}
\newcommand{\Mjup}{\ensuremath{\mathrm{M}_{\mathrm{J}}}}
\newcommand{\kms}{\ensuremath{\mathrm{km}~\mathrm{s}^{-1}}}
\newcommand{\mps}{\ensuremath{\mathrm{m}~\mathrm{s}^{-1}}}
\newcommand{\pc}{\ensuremath{\mathrm{pc}}}
\newcommand{\kpc}{\ensuremath{\mathrm{kpc}}}
\newcommand{\kmskpc}{\ensuremath{\mathrm{km}~\mathrm{s}^{-1}~\mathrm{kpc}^{-1}}}
\newcommand{\dayd}{\ensuremath{\mathrm{d}}}
\newcommand{\yr}{\ensuremath{\mathrm{yr}}}
\newcommand{\AU}{\ensuremath{\mathrm{AU}}}
\newcommand{\Kel}{\ensuremath{\mathrm{K}}}
\newcommand{\mas}{\ensuremath{\mathrm{mas}}}

% Astronomy
\newcommand{\DM}{{\rm DM}}
\newcommand{\abunratio}[2]{\ensuremath{{[\mathrm{#1}/\mathrm{#2}]}}}
\newcommand{\feh}{\abunratio{Fe}{H}}
\newcommand{\mh}{\abunratio{M}{H}}
\newcommand{\alphafe}{\abunratio{\alpha}{Fe}}
\newcommand{\mgfe}{\abunratio{Mg}{Fe}}
\newcommand{\df}{\acronym{DF}}
\newcommand{\logg}{\ensuremath{\log g}}
\newcommand{\Teff}{\ensuremath{T_{\textrm{eff}}}}
\newcommand{\mtwomin}{\ensuremath{M_{2, {\rm min}}}}

% Colors:
\definecolor{tabblue}{HTML}{4E79A7}
\definecolor{taborange}{HTML}{F28E2B}
\definecolor{tabgreen}{HTML}{59A14F}
\definecolor{tabred}{HTML}{E15759}
\definecolor{tabpurple}{HTML}{B07AA1}

% TODO:
\newcommand{\TODO}[1]{{\color{tabgreen}\textbf{TODO:} #1}}
\newcommand{\APWTODO}[1]{{\color{tabpurple}\textbf{APW TODO:} #1}}
\newcommand{\HOGGTODO}[1]{{\color{tabred}\textbf{HOGG TODO:} #1}}


% text macros
\newcommand{\methodname}{\textsl{Orbital Torus Imaging}}

% Project-specific macros - all others go in preamble.tex
\newcommand{\gaia}{\textsl{Gaia}}
\newcommand{\dr}[1]{\acronym{DR}#1}
\newcommand{\apogee}{\acronym{APOGEE}}
\newcommand{\sdss}{\acronym{SDSS}}
\newcommand{\sdssiv}{\acronym{SDSS-IV}}
\newcommand{\sdssv}{\acronym{SDSS-V}}

% trust in Hogg
\setlength{\parindent}{1.1\baselineskip}
\renewcommand{\twocolumngrid}{}
\sloppy\sloppypar\raggedbottom\frenchspacing
\renewcommand{\doi}[1]{{\footnotesize\href{https://doi.org/#1}{#1}}}

\shorttitle{The Outer Galactic Disk with APOGEE}
\shortauthors{Price-Whelan et al.}

\begin{document}
\graphicspath{ {figures/} }
\DeclareGraphicsExtensions{.pdf}

\title{\textbf{%
    Charting the Contorted Outer Disk of the Milky Way with APOGEE}}

\newcommand{\affcca}{Center for Computational Astrophysics, Flatiron Institute, 162 Fifth Ave, New York, NY 10010, USA}
% \newcommand{\affccpp}{Center for Cosmology and Particle Physics, Department of Physics, New York University, 726 Broadway, New York, NY 10003, USA}
% \newcommand{\affmpia}{Max-Planck-Institut f\"ur Astronomie, K\"onigstuhl 17, D-69117 Heidelberg, Germany}
% \newcommand{\affcolumbia}{Department of Astronomy, Columbia University, New York, NY 10027, USA}
% \newcommand{\affutah}{Department of Physics and Astronomy, University of Utah, 115 S. 1400 E., Salt Lake City, UT 84112, USA}
% \newcommand{\affuw}{Department of Astronomy, University of Washington, Box 351580, Seattle, WA 98195, USA}
% \newcommand{\affprinceton}{Department of Astrophysical Sciences, Princeton University, 4 Ivy Lane, Princeton, NJ~08544}
% \newcommand{\affcarnegie}{The Observatories of the Carnegie Institution for Science, 813 Santa Barbara St., Pasadena, CA~91101}

\author[0000-0003-0872-7098]{Adrian~M.~Price-Whelan}
\affiliation{\affcca}

% \author[0000-0003-2866-9403]{David~W.~Hogg}
% \affiliation{\affcca}
% \affiliation{\affmpia}
% \affiliation{\affccpp}

% \author[0000-0001-6244-6727]{Kathryn~V.~Johnston}
% \affiliation{\affcca}
% \affiliation{\affcolumbia}

% \author[0000-0001-5082-6693]{Melissa~K.~Ness}
% \affiliation{\affcolumbia}
% % \affiliation{\affcca}

% \author[0000-0003-4996-9069]{Hans-Walter~Rix}
% \affiliation{\affmpia}

% % APOGEE authors:

% \author[0000-0002-1691-8217]{Rachael L. Beaton}
% \altaffiliation{Carnegie-Princeton Fellow}
% \affiliation{\affprinceton}
% \affiliation{\affcarnegie}

% \author[0000-0002-8725-1069]{Joel~R.~Brownstein}
% \affiliation{\affutah}

% \author[0000-0002-1693-2721]{Domingo~An\'ibal~Garc\'ia-Hern\'andez}
% \affiliation{Instituto de Astrof\'isica de Canarias, 38205 La Laguna, Tenerife, Spain}

% \author[0000-0001-5388-0994]{Sten Hasselquist}
% \altaffiliation{NSF Astronomy and Astrophysics Postdoctoral Fellow}
% \affiliation{\affutah}

% \author[0000-0003-2969-2445]{Christian~R.~Hayes}
% \affiliation{\affuw}

% \author{Richard~R.~Lane}
% \affiliation{Instituto de Astronom\'ia y Ciencias Planetarias de Atacama, Universidad de Atacama, Copayapu 485, Copiap\'o, Chile}

% \author[0000-0001-6761-9359]{Gail~Zasowski}
% \affiliation{\affutah}


\begin{abstract}\noindent
  The outer disk of the Milky Way is significantly distorted.
  The observed flaring of the outer disk, spatially-coherent substructures
  (e.g., the anticenter stream), and recently-observed vertical distortions in
  the positions and velocities of stars in this region all suggest that the
  outer disk has been significantly perturbed from ongoing and past accretion of
  satellite galaxies.
  Here, we map the chemical and kinematic structure of the outer disk using
  element abundances and spectrophotometric distances for red giant stars
  observed by the APOGEE surveys.
  We show that the element abundance distributions of stars in the disk vary
  smoothly as a function of mono-kinematic selections of stars (using actions)
  that extend to radii $R>20~\kpc$.
  We compare observed variations in the bulk vertical motions of stars and the
  (vertical) action distribution of stars with a high-resolution simulation of a
  Milky Way-like disk with a perturbing satellite on a Sagittarius-like orbit.
  We find that the observed vertical distortions and flaring of the outer disk
  can be qualitatively reproduced with this simulation, whereas these features
  cannot be explained by secular evolution of the disk (i.e. radial migration).
  The outer Milky Way disk provides a kinematic fossil record of perturbations
  whose kinematics will enable unique measurements of the dark matter content at
  this radii, and whose chemo-dynamical structures will enable further
  uncovering the accretion history and formation of the Galaxy.
\end{abstract}

\keywords{\raggedright % now from the UAT, not the AAS keyword system
  chemical~abundances % UGH ELEMENTS
  ---
  galaxy~dynamics
  ---
  Milky~Way~dynamics
  ---
  radial~velocity
  ---
  spectroscopy
  ---
  stellar~kinematics
  ---
  surveys
}

% \section*{}\clearpage
\section{Introduction}
\label{sec:intro}

Sup.


\section{Data}
\label{sec:data}

% In our toy demonstrations below, our main data source is a cross-match between
% spectroscopic data from the \apogee\ surveys \citep{Majewski:2017} and
% astrometric data from the \gaia\ mission \citep{Gaia-Collaboration:2016,
% Gaia-Collaboration:2018}.

% \apogee\ is a spectroscopic sub-survey and component of the Sloan Digital Sky
% Survey IV (\sdssiv; \citealt{Eisenstein:2011, Blanton:2017}) whose main goal is
% to map the chemical and dynamical properties of stars across the Milky Way disk.
% The survey uses two nearly identical, high-resolution ($R \sim 22,500$;
% \citealt{Wilson:2019}), infrared ($H$-band) spectrographs---one in the Northern
% hemisphere at Apache Point Observatory (APO) using the SDSS 2.5m telescope
% \citep{Gunn:2006}, and one in the Southern hemisphere at Las Campanas
% Observatory (LCO) using the 2.5m du Pont telescope \citep{Bowen:1973}.
% The primary survey targets are selected with simple color and magnitude cuts
% (\citealt{Zasowski:2013, Zasowski:2017}, Santana et al. in prep., Beaton et al.
% in prep.), but the sparse angular sky coverage and limited number of fibers per
% field lead to a ``pencil-beam''-like sampling of the Milky Way stellar density.
% \apogee\ spectra are reduced \citep{Nidever:2015} and then analyzed (i.e., to
% measure stellar parameters and abundances) using the \apogee\ Stellar Parameters
% and Chemical Abundance Pipeline (\acronym{ASPCAP}; \citealt{ASPCAP,
% Holtzman:2018, Jonsson:2020}); here we use abundance measurements from the
% standard \apogee\ pipeline.

% Here we use a recent internal data product (which includes all data taken
% through March 2020) from the \apogee\ surveys (post-\dr{16}) that includes
% $\approx60\%$ more stars than the publicly-available \dr{16} catalogs
% \citep{DR16, Jonsson:2020}, but was reduced and processed using the same
% pipeline used to produce the \dr{16} release \citep[i.e., the pipeline described
% in][]{Jonsson:2020}.
% This \apogee\ catalog contains calibrated element abundance measurements for 18
% elements, but these have a variety of physical origins and a range of
% reliabilities and measurement precisions.
% For demonstrations below, we therefore focus on a subset of eight, well-measured
% (log) abundance ratios selected to have varied astrophysical origins:
% \abunratio{Fe}{H}, \abunratio{C}{Fe}, \abunratio{N}{Fe}, \abunratio{O}{Fe},
% \abunratio{Mg}{Fe}, \abunratio{Si}{Fe}, \abunratio{Mn}{Fe}, \abunratio{Ni}{Fe}.
% In some cases, we focus on just a single element abundance, \abunratio{Mg}{Fe},
% which is one of the most precisely and accurately determined element abundance
% measured with the \dr{16} pipeline \citep{Jonsson:2020}.

% \gaia\ is primarily an astrometric mission and survey
% \citep{Gaia-Collaboration:2016} that obtains sky position, proper motion, and
% parallax measurements for $>1$ billion stars, limited only by their apparent
% magnitudes (\gaia\ $G \lesssim 20.7$).
% Here we use parallax and proper motion measurements released in \gaia\ \dr{2}
% \citep{Gaia-Collaboration:2018, Gaia-astrometric:2018}.

Yo.

% \begin{figure}[!tp]
%   \begin{mdframed}
%   \color{captiongray}
%   \begin{center}
%   \includegraphics[width=\textwidth]{apogee-rgb-loalpha-mh-am-xy.pdf}
%   \end{center}
%   \caption{%
%     TODO
%     \label{fig:mh-am-xy}}
%   \end{mdframed}
% \end{figure}

\section{Methods}
\label{sec:methods}

\subsection{Computing Actions}
\label{sec:est-actions}



\section{A Tour of the Outer Disk}
\label{sec:tour}

\subsection{Position-based Selections}
\label{sec:sel-positions}

\subsection{Action-based Selections}
\label{sec:sel-actions}



\section{Results}
\label{sec:results}

\subsection{Revisiting Known Substructures}
\label{sec:triandplus}

Figure: some plot of Vphi, VR, Vz vs. Rg or R, show all stars + these stars

TriAnd, A13, GASS/Monoceros. Chemically consistent with low-alpha disk. Kinematics consistent with what the disk is doing out there, albeit f'ed!

ACS??

\subsection{Bulk Motion}
\label{sec:bulkmotion}

Bulk vertical and radial motion as a funtion of R or Rg. Same as what Antoja sees.

\subsection{Radial Migration?}
\label{sec:radialmigration}

Look at Jz distribution, check against radial migration models? But need to factor in selection function.



\section{Comparison to Simulations}
\label{sec:sims}

\subsection{Sagittarius + Milky Way}
\label{sec:sag-only}

\subsection{Milky Way + Secular Effects}
\label{sec:mw-secular}

Sellwood sim?? Or qualitative comparison to expectations extracted from those simulations?


\section{Discussion}
\label{sec:discussion}



\section{Conclusions}
\label{sec:conclusions}



\acknowledgments
It is a pleasure to thank
  ...people.

Funding for the Sloan Digital Sky
Survey IV has been provided by the
Alfred P. Sloan Foundation, the U.S.
Department of Energy Office of
Science, and the Participating
Institutions.

SDSS-IV acknowledges support and
resources from the Center for High
Performance Computing  at the
University of Utah. The SDSS
website is www.sdss.org.

SDSS-IV is managed by the
Astrophysical Research Consortium
for the Participating Institutions
of the SDSS Collaboration including
the Brazilian Participation Group,
the Carnegie Institution for Science,
Carnegie Mellon University, Center for
Astrophysics | Harvard \&
Smithsonian, the Chilean Participation
Group, the French Participation Group,
Instituto de Astrof\'isica de
Canarias, The Johns Hopkins
University, Kavli Institute for the
Physics and Mathematics of the
Universe (IPMU) / University of
Tokyo, the Korean Participation Group,
Lawrence Berkeley National Laboratory,
Leibniz Institut f\"ur Astrophysik
Potsdam (AIP),  Max-Planck-Institut
f\"ur Astronomie (MPIA Heidelberg),
Max-Planck-Institut f\"ur
Astrophysik (MPA Garching),
Max-Planck-Institut f\"ur
Extraterrestrische Physik (MPE),
National Astronomical Observatories of
China, New Mexico State University,
New York University, University of
Notre Dame, Observat\'ario
Nacional / MCTI, The Ohio State
University, Pennsylvania State
University, Shanghai
Astronomical Observatory, United
Kingdom Participation Group,
Universidad Nacional Aut\'onoma
de M\'exico, University of Arizona,
University of Colorado Boulder,
University of Oxford, University of
Portsmouth, University of Utah,
University of Virginia, University
of Washington, University of
Wisconsin, Vanderbilt University,
and Yale University.

This work has made use of data from the European Space Agency (\acronym{ESA})
mission \gaia\ (\url{https://www.cosmos.esa.int/gaia}), processed by the \gaia\
Data Processing and Analysis Consortium (\acronym{DPAC},
\url{https://www.cosmos.esa.int/web/gaia/dpac/consortium}). Funding for the
\acronym{DPAC}
has been provided by national institutions, in particular the institutions
participating in the \gaia\ Multilateral Agreement.

% \facilities{
% \sdss-iv,
% \apogee,
% \gaia
% }

\software{
  \package{Astropy} \citep{astropy, astropy:2018},
  \package{gala} \citep{gala},
  \package{IPython} \citep{ipython},
  \package{matplotlib} \citep{Hunter:2007},
  \package{numpy} \citep{numpy:2020},
  \package{schwimmbad} \citep{schwimmbad},
  \package{scipy} \citep{Virtanen:2020}.
}

% \appendix

% \section{Re-weighting the $K$ nearest neighbors to account for steep gradients}
% \label{app:knn-weights}


\bibliographystyle{aasjournal}
\bibliography{outerdisk}

\end{document}
